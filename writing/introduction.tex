\chapter{Introduction}
\label{sec:org06668a7}
OUTLINE:
    * Cryptocurrency
    * INtroduced by Satoshi Nakimoto
    * A future of the currencies of financial markets. 
    * Doubts if this may be possible.
    * Viewed more as a speculative asset.
    * Due to huge volatility.
    * And large jumps.
    * Volatility is an important measure in finance.
    * Volatility in a high frequency setting.
    * Jumps in volatility 
    * Volatility is non-linear
    * Most models used are often linear.
    * The success of machine learning, and especially deep learning.
    * Gives availability to non-linear models.
    * But the deep learning models may not be fit for time series data and inference.
    * How if we combine the two - can we use the neural networks to improve the econometric model's prediction accuracy and the econometric models to improve the inference of the neural networks?





The thesis will be structured as follows.
Chapter 2 will contain an examination of the financial theory underlying volatility forecasting and cryptocurrency trading.
The linear econometric time series models and their assumptions will be introduced in chapter 3.
Chapter 4 is devoted to the field of deep learning, with an overview of the types of neural networks and their methods used in the analysis.
The evaluation setup of the analysis will be presented in chapter 5.
Chapter 6 will present the results of the analysis, with a comparison between the econometric models as baselines and the neural network.
The discussion, chapter 7, will be revolving around the applicability of neural networks for volatility forecasting and especially the ability of making inference on neural networks.
Chapter 8 will conclude the project

\chapter{Research Question}
\label{sec:org6232157}
\emph{How can neural networks enhance the forecasting capability of HAR-style models in realized volatility}

